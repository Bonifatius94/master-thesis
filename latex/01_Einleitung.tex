
\cleardoublepage
\chapter{Einleitung}\label{ch:Einleitung}

Aufgrund der um mehrere Magnituden potenteren Rechenkapazitäten moderner Hardware
finden Tiefe Neuronale Netze in vielerlei Gebieten des täglichen Lebens breite
Anwendung. Unter anderem werden Neuronale Netze in der Bild- und Spracherkennung
und beim Übersetzen und Generieren von Texten und Bildern eingesetzt.
Auch für komplexe Aufgaben der Mess- und Regelungstechnik wie z.B. die autonome
Steuerung von Fahrzeugen sind Neuronale Netze hervorragend geeignet. Mithilfe von
Methoden des Bestärkenden Lernens (engl. Reinforcement Learning) können interessante
Verhaltensweisen in Simulationen erlernt und anschließend in der Realität erprobt werden.
Für die autonome Fortbewegung der Zukunft ist neben klassischen Automobilen vor allem
die Mikromobilitätsklasse mit E-Scootern oder kleinen Lieferrobotern aufgrund ihrer
vielseitigen Anwendungsmöglichkeiten interessant. Wegen der anspruchsvollen Interaktion
mit Fußgängern ist aktuell das Fahren von E-Scootern in Fußgängerzonen und auf Gehwegen laut
StVO verboten \cite{ekfv2019}. Da jedoch auf vielen Werksgeländen und in der Lagerlogistik
sehr wohl autonome Fahrzeuge in fußgängerbelebten Zonen zum Einsatz kommen \cite{ivm2019},
scheint eine breite Erprobung von Mikromobilitätsfahrzeugen in Fußgängerzonen und auf
Gehwegen nur eine Frage der Zeit zu sein. Die Entwicklung autonomer Fahrsoftware kann
die sichere Nutzung von Mikromobilität entscheidend vorantreiben.\\

Um die Herangehensweise an die Entwicklung autonomer Fahrsoftware besser zu verstehen,
muss zunächst das Gebiet das Autonomen Fahrens näher betrachtet werden. Es handelt sich
im Wesentlichen um die beiden Aufgabenbereiche der Lokalen bzw. Globalen Navigation.
Hierbei entspricht die Globale Navigation einem Routenplaner wie beispielsweise einem
Navigationsgerät, das kürzeste Wege vom aktuellen Standpunkt zum Zielort berechnet.
Hingegen fallen in den Aufgabenbereich der Lokalen Navigation alle Vorgänge, die sonst
ein menschlicher Fahrer übernommen hätte. Eine Fahrsoftware muss demnach das Fahrzeug
beschleunigen bzw. abbremsen und lenken. Zudem beobachtet sie die anderen Verkehrsteilnehmer
und schätzt deren Bewegungen ein, um Kollisionen zu vermeiden. Des weiteren müssen Verkehrszeichen
und Ampeln erkannt und hinsichtlich der Einhaltung der Verkehrsregeln interpretiert werden.
Da die Globale Navigation anhand von Kartenmaterial bereits sehr gut erforscht ist und
effiziente Navigationsalgorithmen zum Finden kürzester Wege z.B. in Form von Heuristiken
wie dem A* Algorithmus zur Verfügung stehen, ist vor allem das Feld der Lokalen
Navigation mit Neuronalen Netzen zur Verarbeitung hochdimensionaler Sensordaten
interessant.\\

Zur Entwicklung autonomer Fahrsoftware existieren bereits zahlreiche Forschungsprojekte,
unter anderem die Ansätze \emph{RobotSF} \cite{machines11020268} und \emph{Multi-Robot}
\cite{fan2020distributed}, \cite{Shunyi2020multirobot2}. Vor allem \emph{RobotSF} ist
interessant, da bei diesem Ansatz Fußgänger durch das Social Force Modell
\cite{helbig1995socialforce}, \cite{moussaid2010groupssf} gesteuert werden und mit einem
steuerbaren Fahrzeug interagieren. Aufgrund der unakzeptabel hohen Kollisionsraten mit
Fußgängern von 13\% bei niedriger und 47\% bei hoher Verkehrsdichte ist der Ansatz jedoch noch
stark verbesserungsfähig. Daher wurde zu Beginn dieser Arbeit der Austausch mit den Forschern
aus Triest gesucht, die hinter der Entwicklung von \emph{RobotSF} stehen. Es wurde berichtet,
dass die Lernexperimente vermutlich noch Fortschritte gezeigt hätten, aber aufgrund der sehr
langen Trainingszeiten von teilweise über einem Monat kaum durchführbar waren und vorzeitig
abgebrochen werden mussten. Um das Experimentieren mit möglichst vielen Ansätzen im Rahmen
dieser und folgender Arbeiten zu ermöglichen, konzentriert sich diese Arbeit neben der
Verbesserung der Verkehrssicherheit deshalb auch auf die Verbesserung der Effizienz der
Simulationsumgebung. Dies umfasst sowohl die Optimierung der Simulationsberechnungen als
auch den Einsatz effektiverer Trainingsverfahren wie Proximal Policy Optimization
\cite{schulman2018ppo}. Dadurch sind schon nach Simulations- bzw. Trainingszeiten
von wenigen Stunden sehr gute Ergebnisse erwartbar.\\

Es soll zunächst die effiziente Erlernbarkeit sicherer Fahrverhaltensweisen bezüglich
der Lokalen Navigation von Mikromobilitätsfahrzeugen mittels Reinforcement Learning
demonstriert werden. Anschließend wird die Qualität der erlernten Verhaltensweisen anhand
von geeigneten Unfallmetriken evaluiert und mit ähnlichen Arbeiten verglichen.


\cleardoublepage
\chapter{Zusammenfassung und Ausblick}\label{ch:Summary}

\section{Zusammenfassung}
Im Rahmen dieser Arbeit wurde eine Fahrsoftware zur Lokalen Navigation
von Mikromobilitätsfahrzeugen mittels Deep Reinforcement Learning entwickelt.
Dabei fokussierten sich die Untersuchungen auf die sichere Interaktion des Fahrzeugs
mit Fußgängern in Fußgängerzonen und auf Gehwegen, was durch die Erweiterung eines
existierenden Simulators \emph{RobotSF} umgesetzt werden konnte. Die 19-fache
Beschleunigung der Simulationszeit ermöglichte die Durchführbarkeit zahlreicher
Lernexperimente auf dem virtuellen Universitätscampus. Zusätzlich konnte durch das
gewählte Lernverfahren der Proximal Policy Optimization eine weitere Effizienzsteigerung
erzielt werden. In Kombination ermöglichten die Verbesserungen bezüglich der Effizienz,
dass Trainingsläufe nun nicht mehr länger als einen Tag daueren, was im Vergleich zur
vorherigen Umsetzung von Caruso et. al mit Trainingszeiten von bis zu einem Monat einen
großen Fortschritt darstellt.\\

Die im adaptierten \emph{RobotSF} Simulator durchgeführten Lernexperimente ergaben
wertvolle Erkenntnisse bezüglich der Erlernbarkeit unfallsicherer Fahrverhaltensweisen.
Durch Anpassungen im Kartenmaterial konnte sehr robustes, sicheres Verhalten in dichten
Menschenmassen erlernt werden. Es wurde zudem deutlich, dass die Gestaltung der Trainingsumgebung
anhand der Zusammenstellung des Kartenmaterials, der Konfiguration der Social Forces und
der gewählten Belohnungsstruktur großen Einfluss auf die erlernten Fahrverhaltensweisen hat.
Unter anderem konnte die Verkehrssicherheit bei dichtem Verkehr durch eine defensive Strategie
im Vergleich zu vorherigen Ansätzen erheblich verbessert werden. Die Kollisionsrate mit Fußgängern
wurde von vormals 47\% auf 0\% gesenkt. Hierbei mussten keinerlei Kompromisse bei der
Ankunftsrate hingenommen werden, da die defensive Strategie eine ähnlich hohe Ankunftsrate
wie vergleichbare Agenten von Caruso et. al aufwies.\\

Die Repräsentation guter Agenten mittels kleiner Modelle mit weniger als einer Million Gewichten
demonstrierte eindrücklich, dass die Trainingszeiten aufgrund der effizient berechenbaren
Modellvorhersagen durch das Sammeln der Trainingsdaten dominiert werden. Hinsichtlich der
Erhöhung der Trainingsdateneffizienz wurde die Optimierung der Trainingseinstellungen
untersucht, wobei die meisten Parameter aus der Konzeption der Modell- und Belohnungsstrukturen
bestätigt und einige weitere Parameter verbessert werden konnten.
Vor allem die einfachere Belohnungsstruktur ermöglichte das Erlernen vielfältiger
Verhaltensweisen und greift nicht zu stark in die Lösungsfindung des Agenten ein, was eine
Verbesserung gegenüber vergleichbaren Arbeiten darstellt.\\

Zusammenfassend kann gesagt werden, dass die Ziele dieser Arbeit für den Anwendungsfall
der sicheren Lokalen Navigation in Fußgängerzonen und auf Gehwegen erreicht wurden.
Außerdem wurden einige interessante Ansätze bezüglich der Entwicklung sicherer Fahrsoftware
mittels Neuronaler Netze präsentiert, die weiter verfolgt werden können. Es ist klar,
dass es bis zur Entwicklung einer Fahrsoftware, die im normalen Straßenverkehr eingesetzt
werden kann, noch ein weiter Weg ist.

\section{Ausblick}
Aufgrund der guten Eigenschaften der erlernten Fahrsoftware ist mittelfristig eine
Erprobung in der Realität mit einem echten Fahrzeug denkbar. Hierfür bietet sich
der Campus der Universität Augsburg an, da die Evaluation der trainierten Fahrsoftware
ohnehin auf dem virtuellen Modell des Campus durchgeführt wurde.\\

Durch das Hinzufügen zusätzlicher Verkehrsteilnehmer, Fahrzeugkinematiken und Entitäten
zur Simulationsumgebung kann die Fahrsoftware in noch realistischeren Szenarien
außerhalb von Gehwegen und Fußgängerzonen erprobt werden. Möglichkeiten für die
Erweiterung wurden aufgezeigt. Zudem kann auch eine weitere Effizienzsteigerung der
Simulationslogik mittels lokalitätsaffiner Datenstrukturen und Algorithmen interessant
sein, um noch mehr Experimente durchführen zu können. Auch die gleichzeitige Simulation
mehrerer Fahrzeuge in derselben Umgebung verspricht zusätzliche Effizienzsteigerungen
und eine Verbesserung der erlernten Fahrqualität.\\

Durch die Auslegung von Simulationsumgebung und Fahrsoftware auf Falsifizierbarkeit
können nun in folgenden Arbeiten die Schwachstellen der trainierten Modelle
identifiziert und behoben werden, um eine sichere Erprobung der Fahrsoftware in der
Realität voranzutreiben.

%   pagestyle.tex 
%   Version 1.0     |   Peter Krönes    |   08.05.2018

%%%%%%%%%%%%%%%%%%%%%%%  PAGESTYLE  %%%%%%%%%%%%%%%%%%%%%%%%%%%%%%%%%%%%

\usepackage{fancyhdr}
\pagestyle{fancy}
\renewcommand{\chaptermark}[1]{\markboth{ 
\thechapter. #1}{}}
\renewcommand{\sectionmark}[1]{\markboth{ 
\thesection. #1}{}}
\fancypagestyle{scrheadings}{
   \fancyhead[LE,RO]{\pagemark}
   \fancyhead[LO]{\nouppercase\leftmark}
   \fancyhead[RE]{\nouppercase\leftmark}
   \fancyfoot[C]{}
}

%%%%%%%%%%%%%%%%%%%%%%%%%%%%%%%%%%%%%%%%%%%%%%%%%%%%%%%%%%%%%%%%%%%%%%%%
% Pagestyle für die ersten Seiten von neuen Kapiteln. Hier wird automatisch Plain ausgewählt wenn fancyhdr benutzt wird.
\fancypagestyle{plain}{%
	\fancyhf{}
	\renewcommand{\headrulewidth}{0pt}
	\fancyhead[LE,RO]{\pagemark}%Seitenzahl oben-rechts im einseitigen Dokument
}
%%%%%%%%%%%%%%%%%%%%%%%%%%%%%%%%%%%%%%%%%%%%%%%%%%%%%%%%%%%%%%%%%%%%%%%%
% Pagestyle für die erste Seite der List of Figures/Tables
\fancypagestyle{pagenumberstyle}{
\renewcommand{\headrulewidth}{0pt}
\fancyhead[LE,RO]{\pagemark} %Seitenzahl oben-rechts
\fancyhead[L]{} %kein Kapitelname oben-links
\fancyfoot[C]{} %keine Seitenzahl unten
}
% Unterscheidung zwischen einseitig und zweiseitigem Druck nicht nötig, da Kapitel bei zweiseitigem Druck immer rechts anfangen.
%%%%%%%%%%%%%%%%%%%%%%%%%%%%%%%%%%%%%%%%%%%%%%%%%%%%%%%%%%%%%%%%%%%%%%%%

%%%%%%%%%%%%%%%%%%%%%%%%%%%%%%%%%%%%%%%%%%%%%%%%%%%%%%%%%%%%%%%%%%%%%%%%
% Pagestyle für scrartcl
\fancypagestyle{articlestyle}{
\renewcommand{\headrulewidth}{0.1pt}
\fancyhead[R]{\pagemark} %Seitenzahl oben-rechts
\fancyhead[L]{} %kein Kapitelname oben-links
\fancyfoot[C]{} %keine Seitenzahl unten
}
% Unterscheidung zwischen einseitig und zweiseitigem Druck nicht nötig, da Kapitel bei zweiseitigem Druck immer rechts anfangen.
%%%%%%%%%%%%%%%%%%%%%%%%%%%%%%%%%%%%%%%%%%%%%%%%%%%%%%%%%%%%%%%%%%%%%%%%

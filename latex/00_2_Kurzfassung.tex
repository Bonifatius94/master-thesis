\chapter*{Kurzfassung}

Diese Arbeit zeigt, wie Mikromobilitätsfahrzeuge, z.B. E-Scooter, die sichere Navigation
durch belebte Zonen mittels autonomer Fahrsoftware erlernen können. Da entsprechende
Fahrzeuge unter anderem auf dem Gehweg oder in Fußgängerzonen unterwegs sind, wird
zunächst eine geeignete Simulationsumgebung für die Interaktion zwischen Fahrzeug und
Fußgängern anhand des Social Force Modells entwickelt. Darauf aufbauend werden
Fahrverhaltensweisen mittels Deep Reinforcement Learning trainiert. Anhand von
Unfallmetriken wird abschließend eine Qualitätssicherung des erlernten Fahrverhaltens
auf Kartenmaterial des virtuellen Universitätscampus durchgeführt und ein Vergleich
der Unfallzahlen mit ähnlichen Projekten aus der Literatur vorgenommen.\\

Unter anderem kann der bestehende Ansatz \emph{RobotSF} \cite{machines11020268} um die
Simulation von Fußgängerzonen und Gehwegen erweitert werden, wobei eine 19-fache
Beschleunigung der für die Simulationslogik benötigten Rechenzeit erzielt wird. Durch die
verbesserte Effizienz ist es möglich, echtes Kartenmaterial während des Trainings und
der Evaluation zu verwenden. Zudem führen die Lernexperimente mit einer deutlich
vereinfachten Belohnungsstruktur zu vielen, interessanten Fahrverhaltensweisen, die Kollisionen
mit Fußgängern in dichten Menschenmengen durch das Erlernen einer defensiven Strategie komplett
vermeiden können.

\cleardoublepage

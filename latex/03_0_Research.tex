
\cleardoublepage
\chapter{Analyse bestehender Ansätze}\label{ch:Research}

In diesem Kapitel werden bereits existierende Ansätze zum Erlernen einer Fahrsoftware
für Mikromobilitätsfahrzeuge beschrieben. Speziell geht es einerseits um den Ansatz
\emph{RobotSF} \cite{machines11020268}, bei dem ein Fahrzeug mit Fußgängern interagiert,
die per Social Force simuliert werden. Alternativ gibt es auch \emph{MultiRobot} Ansätze
\cite{fan2020distributed}, \cite{Shunyi2020multirobot2}, die stattdessen viele
autonome Fahrzeuge miteinander interagieren lassen, anstatt die beweglichen Hindernisse
durch Fußgänger zu simulieren.

\section{Ansätze bezüglich des Lernverfahrens}
Gemeinsam haben die Ansätze, dass meist ein oder mehrere Roboter der Mikromobilitätsklasse
per Deep Reinforcement Learning trainiert werden. Als Trainingsalgorithmen kommen oftmals
Policy Gradient Methoden wie beispielsweise Asynchronous Advantage Actor-Critic (A3C) oder
Proximal Policy Optimization (PPO) zu Einsatz, da diese gut mit Parallelität und
Rechenressourcen skalieren. Im Fall von rechenintensiven Simulationen können auch
Lernverfahren wie Deep Deterministic Policy Gradient (DDPG) und Soft Actor-Critic (SAC)
interessant sein, um eine bessere Trainingsdateneffizienz zu erreichen \cite{Kiran2022survey}.

\section{Ansätze bezüglich Fahrzeug und Sensorik}
Bezüglich der verwendeten Fahrzeuge kommen verschiedenste Modelle zum Einsatz. Je nach Fahrzeug
bietet sich eine unterschiedliche Kinematik zur Simulation an. Beispielsweise eignet sich
das Bicycle Model, um E-Scooter zu modellieren. Für Fahrzeuge mit Allradantrieb oder Ketten
kommt Differential Drive infrage. Fan et. al \cite{fan2020distributed} geht sogar so weit,
einen trainierten Agent bewusst mit einer abgewandelten Kinematik zu evaluieren, um die
Robustheit der erlernten Fahrsoftware zu testen. Als Sensorik wird in der Mikromobilität
meist ein LiDAR-Sensor in Kombination mit einer Zielpeilung verwendet, was alle 3 betrachteten
Veröffentlichungen gemeinsam haben.

\section{Ansätze bezüglich der Simulationsumgebung}
Für die Repräsentation der Entitäten und des Kartenmaterials bei der Simulation gibt es auch
vielerlei Ansätze \cite{Kiran2022survey}. Unter anderem werden Entitäten mit Vektorgrafiken
dargestellt. Ein weiterer Ansatz diskretisiert die Karte und alle Entitäten als Raster
(engl. Occupancy Grid). Weitere Ansätze wie CALRA \cite{dosovitskiy2017carla} setzen auf
fotorealistische Grafiken mittels optimierter Spieleframeworks wie der Unreal Engine.
Der jeweilige Ansatz hat großen Einfluss auf die verwendeten Datenstrukturen und
Algorithmen, um die Entitäten und Sensoren zu simulieren.

\section{Ansätze bezüglich der Evaluation}
Zur Evaluation der trainierten Modelle gibt es noch keine einheitlichen Standards
\cite{Kiran2022survey}. Dies macht es sehr schwer, die verschiedenen Ansätze der Forscher
fair zu vergleichen. Eine Vergleichbarkeit ist aber ohnehin kaum herstellbar, da die
Vielfalt bezüglich Fahrzeugen und Sensorik teilweise unüberwindbare Inkompatibilitäten
verursacht. Als Metrik werden meist Erfolgs- bzw. Unfallraten für die relative Häufigkeit
an Zielfahrten bzw. Unfällen verwendet.
